\section{Trabajos Relacionados}
\subsection{DeepChess: End-to-End Deep Neural Network for Automatic Learning in Chess}

Este trabajo describe el desarrollo de DeepChess, un modelo de red neuronal profunda que aprende a evaluar posiciones de ajedrez sin conocimiento previo de las reglas del juego. El entrenamiento se realiza utilizando una gran base de datos de partidas de ajedrez, combinando preentrenamiento no supervisado con entrenamiento supervisado para comparar posiciones y elegir la más favorable. DeepChess es relevante para el desarrollo de un modelo de red neuronal para ajedrez cuántico porque demuestra la viabilidad de entrenar un modelo desde cero, sin reglas predefinidas, lo cual es aplicable a la naturaleza compleja y diferente de las reglas del ajedrez cuántico.\cite{David201688}

\subsection{Monte Carlo Tree Search, Neural Networks \& Chess}

El documento analiza la combinación de Monte Carlo Tree Search (MCTS) y redes neuronales convolucionales (CNN) con la finalidad de crear un motor de ajedrez que pueda competir contra otros motores como Stockfish. Inspirado en AlphaZero, se utiliza MCTS para
explorar posiciones y CNN para evaluar la calidad de las jugadas. Este enfoque es relevante para un modelo de ajedrez cuántico porque MCTS podría ser una herramienta útil para manejar la complejidad de las posibles configuraciones cuánticas, mientras que las CNN pueden ayudar a evaluar posiciones no tradicionales en el tablero.\cite{steinberg2021}

\subsection{Aprendizaje por refuerzo y ajedrez: Alpha Zero}

AlphaZero, es un modelo basado en aprendizaje por refuerzo que se entrena jugando millones de partidas contra sí mismo que utiliza utilizando redes neuronales y búsqueda de Monte Carlo con la finalidad de evaluar posiciones (states) y tomar decisiones sobre cada una de las jugadas que debe realizar. Sin necesidad de depender de datos preexistentes, aprende estrategias avanzadas y no convencionales, optimizando su rendimiento mediante la mejora continua en base a experiencias previas, lo que le permitió superar a programas tradicionales como Stockfish en ajedrez.\cite{solana2022}