\section{Trabajo Relacionado}
\subsection{DeepChess: End-to-End Deep Neural Network for Automatic Learning in Chess}
Este trabajo describe el desarrollo de DeepChess, un modelo de red neuronal profunda
que aprende a evaluar posiciones de ajedrez sin conocimiento previo de las reglas del
juego. El entrenamiento se realiza utilizando una gran base de datos de partidas de
ajedrez, combinando preentrenamiento no supervisado con entrenamiento supervisado
para comparar posiciones y elegir la más favorable. DeepChess es relevante para el
desarrollo de un modelo de red neuronal para ajedrez cuántico porque demuestra la
viabilidad de entrenar un modelo desde cero, sin reglas predefinidas, lo cual es aplicable
a la naturaleza compleja y diferente de las reglas del ajedrez cuántico.\cite{David201688}

\subsection{Neural Networks for Chess: The Magic of Deep and Reinforcement Learning Revealed}

Este documento proporciona una visión detallada de cómo las redes neuronales,
especialmente aquellas basadas en aprendizaje profundo y reforzamiento, han
revolucionado los motores de ajedrez, mencionando casos como AlphaZero y Leela Chess
Zero que se destacan por su enfoque de autoaprendizaje y búsqueda mediante Monte
Carlo. La información sobre AlphaZero y su enfoque de autoaprendizaje puede ser crucial
para desarrollar un modelo de ajedrez cuántico, ya que permite entender cómo un
sistema puede aprender estrategias complejas por sí mismo, similar a lo que se
necesitaría para un juego con reglas cuánticas.\cite{klein2021neural}

\subsection{Monte Carlo Tree Search, Neural Networks \& Chess}

El documento analiza la combinación de Monte Carlo Tree Search (MCTS) y redes
neuronales convolucionales (CNN) para crear un motor de ajedrez que pueda competir
contra otros motores como Stockfish. Inspirado en AlphaZero, se utiliza MCTS para
explorar posiciones y CNN para evaluar la calidad de las jugadas. Este enfoque es
relevante para un modelo de ajedrez cuántico porque MCTS podría ser una herramienta
útil para manejar la complejidad de las posibles configuraciones cuánticas, mientras que
las CNN pueden ayudar a evaluar posiciones no tradicionales en el tablero.\cite{steinberg2021}

\subsection{Predicting Chess Moves with Multilayer Perceptron and Limited Lookahead}

Este estudio explora el uso de una red neuronal perceptrón multicapa para predecir
movimientos de ajedrez sin una búsqueda profunda. El enfoque se centra en evaluar el
tablero de manera eficiente para reducir la complejidad de las búsquedas. Es útil para un
modelo de ajedrez cuántico ya que proporciona una forma de evaluar las posiciones del
tablero de manera rápida y eficiente, lo cual es crucial dado el aumento de posibles
estados en un sistema cuántico.\cite{mehta2020predicting}

\subsection{Training a Convolutional Neural Network to Evaluate Chess Positions}

Este trabajo investiga la capacidad de una red neuronal convolucional para evaluar
posiciones de ajedrez al predecir las evaluaciones del motor Stockfish. Utiliza datos de
partidas de ajedrez para entrenar el modelo y demuestra que las CNN pueden captar
características complejas de las posiciones en el tablero. La aplicación de CNN para
evaluar posiciones es relevante para el ajedrez cuántico, ya que permite el análisis de
configuraciones complejas, lo cual es fundamental para entender las dinámicas del juego
cuántico.\cite{vikstrom2019training}