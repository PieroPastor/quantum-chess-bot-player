%%%%%%%%%%%%%%%%%%%%%%%%%%%%%%%%%%%%%%%%%%%%
% PAQUETES
%%%%%%%%%%%%%%%%%%%%%%%%%%%%%%%%%%%%%%%%%%%%
% Paquetes para que soporte caracteres especiales
\usepackage[spanish]{babel}
\usepackage[utf8]{inputenc}
\usepackage[T1]{fontenc}
%
\usepackage[hidelinks]{hyperref} % links internos
\usepackage{multicol} % multiples columnas
\usepackage{xcolor} % colores
\usepackage{graphicx} % para insertar imagenes
\usepackage{import} % para insertar imagenes
%Podemos poner top=2cm,left2cm o usar la palabra centering
% El showframe es para poder visualizar los margenes del archivo pdf
%\usepackage[total={18cm,23cm},top=2cm,left=2cm,showframe]{geometry}
% 
\usepackage[total={18cm,23cm},top=2cm,left=2cm]{geometry}
\usepackage{amsmath}
\usepackage{tcolorbox} % para cuadros de colors personalizados
% Para manejo de las Citas y trabajar con los archivos bib
\usepackage{cite}  % Para manejar las citas
%\usepackage{biblatex}
% Para el encabecado de las caratulas
\usepackage{fancyhdr}
\usepackage{lipsum}% 
\usepackage{titlesec} % Para personalizar títulos de secciones
\usepackage{tikz} % para los graficos
\usepackage{adjustbox} % para que los graficos ocupen solo cierta dimension
%%%%%%%%%%%%%%%%%%%%%%%%%%%%%%%%%%%%%%%%%%%
% Configuracion de color para las secciones, subsecction y subsubsecction
%%%%%%%%%%%%%%%%%%%%%%%%%%%%%%%%%%%%%%%%%%%
% Aplica el color a las secciones
\titleformat{\section}{\color{pucpVerde}\normalfont\Large\bfseries}{\thesection}{1em}{}
\titleformat{\subsection}{\color{pucpVerde}\normalfont\large\bfseries}{\thesubsection}{1em}{}
\titleformat{\subsubsection}{\color{pucpVerde}\normalfont\normalsize\bfseries}{\thesubsubsection}{1em}{}

%%%%%%%%%%%%%%%%%%%%%%%%%%%%%%%%%%%%%%%%%%%
% Configuracion de FANCYHDR
%%%%%%%%%%%%%%%%%%%%%%%%%%%%%%%%%%%%%%%%%%%
% Encabezado
\pagestyle{fancy}
\fancyhf{}
\setlength{\headheight}{1.5cm}
\fancyhead[L]{\textcolor{pucpVerde}{\textsf{INTELIGENCIA ARTIFICIAL (1INF24)}}} % Texto en la izquierda
\fancyhead[R]{
    \raisebox{-0.5\height}{\resizebox{!}{1cm}{\import{./Imagenes/}{pucp_logo.pdf_tex}}} % Ajuste de la altura a 2cm
} % Logo a la derecha

% Línea debajo del encabezado
%\renewcommand{\headrulewidth}{0.4pt}

% Personalizar la línea debajo del encabezado
\renewcommand{\headrulewidth}{1pt}  % Cambia el grosor a 1.5pt
\renewcommand{\headrule}{\hbox to\headwidth{\color{pucpColor}\leaders\hrule height \headrulewidth\hfill}}  % Cambia el color a azul
